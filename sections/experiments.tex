\section{Experiments}

\subsection{Datasets}

\subsubsection{Experimental datasets}

To our knowledge, there's no existing large scale database of annotated X-ray
single particle images categorized by biological/chemical samples.  The Coherent
X-ray Data Bank (CXIDB) serves as ``a permanent public repository of data from
coherent X-ray sources" \cite{maiaCoherentXrayImaging2012}.  Table \ref{tb: SPI
experiments} is a list of all SPI experiments documented on CXIDB.  We decide to
use the raw data collected in experiment amo06516
\cite{liDiffractionDataAerosolized2020}, part of which are also available by
checking out CXIDB ID 156.  One major difference between raw data and deposited
data is that raw data contain a considerable amount of \textit{non-sample-hit}
images, which are important for real-time SPI classification tasks.  For data
labelling, we created a GUI tool (\url{https:
//github.com/carbonscott/hit-labeler}) that can label hits from multiple sources,
including raw data through \textit{psana} \cite{damianiLinacCoherentLight2016}
and HDF5 files from CXIDB.  In total, we labelled {\color{red}xxx}
\textit{single-hit}, xxx \textit{multi-hit} and xxx \textit{non-sample-hit}.  

\begin{table}
    \caption{All SPI experiments documented on CXIDB.}
    \label{tb: SPI experiments}
    %% \renewcommand{\arraystretch}{1.2}
    %% \resizebox{1.0\textwidth}{!}{
        \begin{tabularx}{\textwidth}{ l l X }
            CXIDB ID & Light Source & Sample \\
            \hline
            1        & LCLS         & Mimivirus                                                       \\
            2        & LCLS         & Mimivirus                                                       \\
            3        & FLASH        & FIB etched 20-nm-thick silicon nitride membrane                 \\
            4-8      & ALS          & Gold labeled frozen dried Saccharomyces cerevisiae yeast cells  \\
            9        & FLASH        & Iron Oxide Ellipsoids                                           \\
            10       & LCLS         & Nanorice                                                        \\
            11       & LCLS         & Magnetosomes                                                    \\
            12       & LCLS         & Tobacco mosaic virus                                            \\
            13       & LCLS         & T4 bacteriophage                                                \\
            14       & LCLS         & Paramecium bursaria Chlorella virus                             \\
            19       & LCLS         & Airborne Particulate Matter (Soot)                              \\
            20       & LCLS         & Clusters of Polystyrene Spheres                                 \\
            25       & LCLS         & Carboxysomes                                                    \\
            26       & LCLS         & Cyanobium gracile                                               \\
            27       & LCLS         & Synechococcus elongatus                                         \\
            28       & ALS          & 50 nm colloidal gold particles                                  \\
            30       & LCLS         & Mimivirus                                                       \\
            36       & LCLS         & Rice Dwarf Virus                                                \\
            37       & LCLS         & Cyanobium gracile and Synechococcus elongtatus                  \\
            56       & LCLS         & Omono River Virus                                               \\
            57       & LCLS         & Gold core and palladium shell nanoparticles                     \\
            58       & LCLS         & Coliphage PR772                                                 \\
            78       & LCLS         & RNA polymerase II                                               \\
            84       & ESRF         & Gold structure (largest diameter about 1.1 um)                  \\
            88       & LCLS         & PR772                                                           \\
            119      & LCLS         & Sucrose                                                         \\
            146      & FLASH        & Xenon nanoclusters                                              \\
            155      & LCLS         & Melbournevirus                                                  \\
            156      & LCLS         & Coliphage PR772                                                 \\
        \end{tabularx}
    %% }
\end{table}


\subsubsection{Simulated datasets}

There's no existing large scale database of annotated SPI images generated by
physics-based simulation.  To investigate generalizability of our SPI classifier,
it's essential to have a large dataset of simulated SPI images. We resort to
\textit{skopi} \cite{peckSkopiSimulationPackage2021}, a GPU-based program for
simulating diffractive images from noncrystalline biomolecules, for concurrently
simulating high-resolution SPI scattering and providing accurate labels in an
automated manner at scale.  Our primary interest is to explore the model
generalizability on \textit{large} single particles, considering the
state-of-art resolution ever reached in an SPI experiment is still at the
nanometer level ($>$ 10 $nm$) {\color{red} (need to fact-check)} unlike in
protein crystallography that studies biological structures at the angstrom
level.  PDB statistics offers direct insights into PDB data distribution by
molecular weight (\url{https:
//www.rcsb.org/stats/distribution-molecular-weight-structure}).  We focus on
\textit{large} particles with molecular weights over 380 $KDa$.  Fig. \ref{fig:
num atom per bio assem} describes the population frequency of the atom numbers
per biological assembly with each area representing 50 PDB items.  The atom
numbers spread across three orders of magnitudes ($10^4\text{-}10^6$).  96.0\%
of \textit{large} particles have $10^4$ atoms, and only 1.2\% have massive
$10^6$ atom numbers. Simulated datasets are generated by setting the detector
distance at 100.0 $mm$ and photon energy at 1.660 $keV$.  Random rotation is
also applied as a means of data augmentation.  

\begin{figure}
\includegraphics[width=1.0\textwidth,keepaspectratio]
{./figures/num_atom_per_bio_assem.pdf}
\caption{Number of atom per biological assembly. {\color{red} Need to indicate where
those molecular graphs are from.}}
\label{fig: num atom per bio assem}
\end{figure}


\subsection{Performance on experimental data}

We split our experimental dataset into training set, validation set and test
set.  


\begin{table}

    \caption{
        The metadata of four experiments that are used to illustrate the
        limiting factors of model performance from the training data standpoint.  \#
        denotes `number of images'.  The numbers in the `Unique \# $/$ class' column
        follow the order of \textit{non-sample-hit}, \textit{single-hit} and
        \textit{multi-hit}.  \# (Train) and \# (Validation) are the number of images,
        including those generated from data augmentation, used in model training and
        validation, respectively.  The last column tells the fraction of all
        labelled images used in the training set.  
    }
    \label{tb : metadata}
    %% \renewcommand{\arraystretch}{1.2}

    %% \centering
    %% \resizebox{1.0\textwidth}{!}{
        \begin{tabularx}{\linewdith}{ l c c c c }
            Experiment &   $\dfrac{Unique \#}{class}$  &  \# (Train) & \# (Validation) & $\dfrac{\text{\#Training}}{\text{\#Total}}$ \\
            \hline
            1          &   18,40,40             &  2000       & 2000            & 0.25      \\
            2          &   18,40,40             &  4000       & 4000            & 0.25      \\
            3          &   40,40,40             &  2000       & 2000            & 0.50      \\
            4          &   44,80,80             &  2000       & 2000            & 0.50      \\
        \end{tabularx}
    %% }
\end{table}



\begin{table}
    \caption{
        Confusion matrix for each experiment mentioned in Table. \ref{tb :
        metadata}.
    }

    \label{tb: Recall}

    %% \centering
    %% \renewcommand{\arraystretch}{1.2}
    %% \resizebox{1.0\columnwidth}{!}{
        \begin{tabularx}{\linewidth}{ l | X X X X X X X X }
            \textbf{Experiment 1} &  N(A) & S(A) & M(A) & ACC  & PRE  & REC           & SPE  & F1   \\
            \hline
            N(P)                  &  335  & 0    & 10   & 0.99 & 0.97 & 1.00          & 0.98 & 0.99 \\
            S(P)                  &  0    & 333  & 24   & 0.97 & 0.93 & 0.99          & 0.96 & 0.96 \\
            M(P)                  &  0    & 5    & 293  & 0.96 & 0.98 & \textbf{0.90} & 0.99 & 0.94 \\
            \hline
            \textbf{Experiment 2} &  N(A) & S(A)  & M(A) & ACC  & PRE  & REC           & SPE  & F1   \\
            \hline
            N(P)                  &  335  & 1     & 9    & 0.99 & 0.97 & 1.00          & 0.98 & 0.99 \\
            S(P)                  &  0    & 327   & 28   & 0.96 & 0.92 & 0.97          & 0.97 & 0.94 \\
            M(P)                  &  0    & 10    & 290  & 0.95 & 0.97 & \textbf{0.89} & 0.99 & 0.93 \\
            \hline
            \textbf{Experiment 3} &  N(A) & S(A)   & M(A) & ACC  & PRE  & REC           & SPE  & F1   \\
            \hline
            N(P)                  &  331  & 0      & 13   & 0.99 & 0.96 & 1.00          & 0.98 & 0.98 \\
            S(P)                  &  0    & 339    & 19   & 0.97 & 0.95 & 0.98          & 0.97 & 0.96 \\
            M(P)                  &  0    & 6      & 292  & 0.96 & 0.98 & \textbf{0.90} & 0.99 & 0.94 \\
            \hline
            \textbf{Experiment 4} &  N(A) & S(A)   & M(A) & ACC  & PRE  & REC           & SPE  & F1   \\
            \hline
            N(P)                  &  331  & 1      & 7    & 0.99 & 0.98 & 1.00          & 0.99 & 0.99 \\
            S(P)                  &  0    & 340    & 4    & 0.99 & 0.99 & 0.99          & 0.99 & 0.99 \\
            M(P)                  &  0    & 4      & 313  & 0.98 & 0.99 & \textbf{0.97} & 0.99 & 0.98 \\
        \end{tabularx}
    %% }
\end{table}
