\section{Experiments}

{
%% \setlength{\parindent}{4em}
\color{gray} 

%% \indent 
\begin{itemize}

    % AMO beamline of the Linac Coherent Light Source (LCLS) at the SLAC
    % National Accelerator Laboratory.
    \item Datasets.

    \item Data preprocessing (image, panel, mosaic)
    % prove order doesn't matter in pattern recognition, as a prerequisite for
    % mosaic mode.

    \item Data augmentation (for training, standardize, crop, random).

    \item Human baseline performance (game of hit)

    \item Accuracy-wise, our model has outperformed previous models.  

    \item Less number of images to train the model (need to provide more results
    to justify).  

    \item Limitations.

\end{itemize}

}


\subsection{Datasets}

%% Why would I need those datasets?  
%% No standard datasets.  Manual labelling.  
%% Okay, simulated datasets are valuable in exploring the model generalizability. .

% ISSUES: The amo06516 might not be correct.  
To our knowledge, there's no existing large scale database of annotated X-ray
single particle images.  Thankfully, we can access and manually label raw
experimental data, specifically amo06516 \cite{liDiffractionDataAerosolized2020},
collected in LCLS through the Coherent X-ray Imaging Data Bank (CXIDB)
\cite{maiaCoherentXrayImaging2012}.  Meanwhile, it's challenging to
automatically scale up annotations for experimental datasets.  We then resort to
skopi \cite{peckSkopiSimulationPackage2021}, a GPU-based diffractive image
simulator for noncrystalline biomolecules, for concurrently synthesizing
high-resolution scatter patterns and accurate labels in an automated manner at
scale.  Lastly, we aim at delivering a SPI classifer capable of real-time
classification during data collection.  One major constraint in real-time
scenario is that we might not have a complete view of a scattering pattern,
assembled from pixel areas of individual detector panels.  We thus consider
three different modes in dataset curation: (1) Image mode, where information
from all detector panels are consolidated according to their respective physical
locations; (2) Mosiac mode, where information from serveral detector panels are
combined without acknowleding any geometric relations among panels;  (3) Panel
mode, meaning that only pixel contents in a single panel during a exposure is
visiable to our model.  

%% ImageNet
%% \cite{dengImageNetLargescaleHierarchical2009} that provides large scale
%% annotated images, .  

\subsubsection{Experimental datasets}


\subsubsection{Simulated datasets}


\subsubsection{Image mode, panel mode and mosaic mode}


\subsubsection{Data augmentation}
