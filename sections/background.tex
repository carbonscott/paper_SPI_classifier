\section{Related work}

\subsection{Classifying SPI \textit{hit}s}

The task of SPI \textit{hit} classification has been explored for over a decade.
Prior to the first CXIDB data deposition
\cite{seibertSingleMimivirusParticles2011} was available to the public,
\textit{hit} classification were performed by evaluating a cross-correlation of
two SPI images in polar coordinates
\cite{bortelClassificationAveragingRandom2009}.  An alternative classification
approach was based on expectation maximization (EM)
\cite{dempsterMaximumLikelihoodIncomplete1977}, which was later integrated in
the EMC method \cite{lohReconstructionAlgorithmSingleparticle2009}, a widely
used particle reconstruction method for SPI.  Additionally, a combination of
size filtering, feature vectors and PCA was used for achieving a better
\textit{single-hit} classification
\cite{bobkovSortingAlgorithmsSingleparticle2015}.  More recently, artificial
neural network models have become a new avenue for exploring classification
solutions with the advent of capable infrastructures (GPUs, machine learning
frameworks) for model training.  For instance,
\cite{shiEvaluationPerformanceClassification2019} uses a CNN model  and
\cite{ignatenkoClassificationDiffractionPatterns2021} employs off-the-shell YOLO
(You only look once) models \cite{redmonYOLO9000BetterFaster2016,
redmonYOLOv3IncrementalImprovement2018}.  Both models were trained to
performance binary classification for identifying \textit{single-hit}s.  However,
they all used a fully connected layer as the classifier that is directly
attached to a CNN vision module.  This design has been known for its
inefficiency in model training as pointed out in
\cite{schroffFaceNetUnifiedEmbedding2015}.  Notably, the YOLO model in
\cite{ignatenkoClassificationDiffractionPatterns2021} takes more than three
hours to train before it achieves an accuracy of 95\% in predicting
\textit{single-hit}s.  The training time is not clear in
\cite{shiEvaluationPerformanceClassification2019}, but its accuracy is 83.8\%.
In contrast, our model demonstrates high accuracy on experimental data while
keeping the training time short.  


\subsection{Generalizing a classifier}

To our knowledge, there are no solution proposed to generalize a SPI \textit{hit}
classifier.  However, many approaches have been developed to generalize object
recognition in computer vision.  One of the early examples was signature
verification using siamese neural networks
\cite{bromleySignatureVerificationUsing1994}.  The model basically measures the
distance between two signature embeddings generated from two identical neural
networks, which learn to seperate embeddings of different signatures in a latent
space and vice versa.  The idea of siamese networks were later applied in face
recognition/verification \cite{chopraLearningSimilarityMetric2005}.  In the last
decade, another breakthrough in face recognition was demonstrated by the use of
triplet loss in \cite{kochSiameseNeuralNetworks2015a}, where both a positive
image and a negative image are constrated against the same anchor image.  Beyond
signature or face verification, few-shot learning problems also require
generalizing an existing classifier to identifying new categories of things not
seen by the model before.  For example, \cite{vinyalsMatchingNetworksOne2017}
introduced ``matching networks" that learn to query an unlabeled image against
images in a small support set that have new labels.
\cite{snellPrototypicalNetworksFewshot2017} proposed a prototypical network that
learns a metric space where classification is effecetively done by measuring
distances between an unlabeled image and the prototype representation, such as
the average, of each new class.  In our application, we choose to use triplet
loss method to train a SPI \textit{hit} classifier.  In contrast to the
thresholding method in FaceNet which requires tuning an additional
hyperparameter, our model treats an input image as a query item, which is then
subject to some similarity tests against embeddings from each \textit{hit}
class.  This approach is also easy to implement because the number of
\textit{hit} classes is three as opposed to human faces of much more unique
identities in face verification problem.  
