\section{Background (Related work)}

%% Supervised solutions
%% based on neural network models learn from labelled images and provide predicted classification
%% (Shi et al., 2019; Ignatenko et al., 2021).
%%
%% SPI classification is like face recognition, which is why FaceNet idea is applied.  
%% 
%% Few-shot learning??? 
%% Train our model on SPI images simulated using 40+ particles, whose MW is over 380K.  It generalizes to completely new particles of similar MW to 85% accuracy.  
%%
%% PDB distribution by Molecular Weight (MW)
%% https://www.rcsb.org/stats/distribution-molecular-weight-structure

{
\setlength{\parindent}{4em}
\color{gray} 

\indent Characterize related work in addressing the problem raised in the
introduction. Meanwhile, what works have also inspired us to propose our method?  (The
FaceNet paper)

}

\subsection{Annotations for experimental SPI images}

% Non ML, many engineered steps.  
% EM based method
% No standard datasets
% Our annotations include unintended hit and background

Our end-to-end model is trained on manually labelled experimental images without
any filtering or intermediate steps.  The previous machine learning model
\cite{ignatenkoClassificationDiffractionPatterns2021} are trained on datasets
constructed using multi-staged selections. Overall, $18,213$ images are subject
to single-hit classification after size-based screening, and $1,085$ single hit
images are used as ground truth after further filtering using principle
component analysis (PCA) on cross-correlation-based engineered features
\cite{roseSingleparticleImagingSymmetry2018}. Moreover, the dataset is improved
by adding human-labelled SPI images, which gives rise to a total of $1,393$
single-hit images \cite{liDiffractionDataAerosolized2020a}.  We are not aware of
the availability of this dataset to the general public.  Therefore, a practical
way to obtain a dataset of annotated SPI images is to curate it by ourselves.
We use a tweaked version of psocake as a data labeler and we have constructed an
experimental dataset with {\color{red}xxx} single-hit, {\color{red}xxx}.  


\subsection{Generating synthetic SPI images}

% No standard datasets
% We repurpose the simulator for making synthetic datasets.

\subsection{SPI image classification}


\subsection{One-shot learning for object recognition}
