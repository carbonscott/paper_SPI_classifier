\section{Introduction}

% (1) What is SPI and what problem in SPI have I been tackling?  
% - SPI is an important technique and where it stands in the structural biology toolkits.
%   - Room temperature.
%   - No need for crystalline.
%   - One or two sentences to show why it's important.  
% - SPI scattering patterns are like what, and classification is the problem.

Single-particle imaging (SPI) by X-ray free-electron lasers (XFELs) is a
promising method of determining the three-dimensional structure of
noncrystalline biomolecules at room temperature.   In SPI experiments,
femtosecond coherent X-ray beams strike biomolecules sprayed into the beam path,
causing scattering of radiation before destruction of the samples inflicted by
the intense X-rays.  This way of collecting scattering datasets is known as the
'diffraction before destruction' approach
\cite{neutzePotentialBiomolecularImaging2000,
chapmanFemtosecondDiffractiveImaging2006,seibertSingleMimivirusParticles2011,
aquilaLinacCoherentLight2015,reddyCoherentSoftXray2017}.  Scattering patterns in
SPI experiments largely fall into four categories, depending on how X-ray pulses
hit sample particles delivered through jet streams.  Astonishingly, 98\% X-ray
pulses miss their target particles
\cite{shiEvaluationPerformanceClassification2019}, leading to no scattering
pattern identified as a 'no hit'.  When X-ray photons collides with one and only
one sample particle, the emerging scattering pattern is defined as a 'single
hit'. Expectedly, a 'multi hit' happens when X-ray pulses and a cluster of
sample particles intersect. In some cases, X-ray pulses might hit non-biological
objects in delivery medium and those resulting scattering patterns are labeled
as 'unintended hit'.  Among the four categories, only 'single hit' images will
contribute to reconstruction of electron density maps through downstream data
processing, e.g. orientation recovery and phase retrieval.  Therefore, an
efficient SPI scattering pattern classifier is long desired, and building one
for real-time applications still remains to be an elusive task.


% (2) The past work on tackling the problem.  
% ...Previous work on X has addressed with Y
% 
% - unsupervised method
%   - [DONE] typically problem-specific
%   - [DONE] prior knowledge about the location of deisred cluster needs to be known.  
% - supervised method
%   - [DONE] some requires complete retraining.  
%   - [CANCEL] requires fine tune, like the layers of the network.  
%   - [ ] still relies on hit finding for binary classifier, whereas classifier doesn't care.  

Some pioneering works that address the challenge of classifying SPI scattering
pattern have been developed.  Unsupervised methods
\cite{yoonUnsupervisedClassificationSingleparticle2011,
giannakisSymmetriesImageFormation2012,schwanderSymmetriesImageFormation2012,
yoonNovelAlgorithmsCoherent2012,
andreassonAutomatedIdentificationClassification2014,
bobkovSortingAlgorithmsSingleparticle2015a} focus on extracting features from
image samples and finding clusters of 'single hit' images.  However,
unsupervised solutions are highly problem-specific, that is to say, it's not
realistic to expect 'single hit' images from different biological samples would
cluster in the feature space.  Moreover, despite being unsupervised, those
solutions also require prior knowledge of data distribution in feature space for
classification, rendering it incapable of being automated. Supervised solutions
based on neural network models \cite{shiEvaluationPerformanceClassification2019,
ignatenkoClassificationDiffractionPatterns2021} have been developed to learn
from labelled images and provide predicted classification.  In those networks,
convolution neural networks (CNN) were employed for feature extraction and a
fully connected layer is directly attached to the convolutional layers for
classification.  Training such networks demands a resonably large sample size
{\color{red}(check those papers for exact numbers)} for each class, which might
not be available during the course of data collection. Additionally, the
population of images vary substantially across different classes, resulting in
imbalanced training dataset can be detrimental to model training.  


% (3) What are the core challenges?  In this work, we do this.  Concretely, our
% contributions are ... 

In this work, we strive to approach the goal of classifying scattering patterns
in real-time data collection.  The primary challenge in a real-time scenario is
that previously proposed models require many manually labelled images on the fly,
which is a unsolved classification problem itself.  It's a known bottleneck in
classification applications, such as in facial recognition.  Additionally, the
model should be trainable prior to real-time data collections instead of having
to be trained on the fly. The second challenge is the lack of datasets that are
sophisticated enough to enable model training. The datasets should have
scattering patterns from different noncrystalline biomolecules.  


% (4) What we will present to tackle the challenge?
% Concrete contributions:
% - Curate a sophisticated datasets that enable training a generalized model.  
% - Train a generalized model for real-time classification.  

Our insight is to turn the classification problem into a one-shot learning
problem, that is to say, rather than trying to classify an input image, we
simply compare similarity between the input image and a set of reference images
from each class.  The class of the most similar reference image would define the
class of the input image in prediction. Concretely, our contributions can be
summerized below:

\begin{itemize}

    \item We train an embedding model on a large self-curated dataset through
    siamese networks and obtain a generalized model for real-time scattering
    pattern classification.  We report an accuracy of {\color{red} xxx} and
    human baseline established by {\color{red} xxx}.

    \item We introduce a new large-scale synthetic SPI image database.  The
    datasets have accurate labels and diverse scattering patterns as they are
    provided by simulating SPI images from {\color{red} 7000+} biomolecules with
    molecular weights larger than 380 $KDa$.  

\end{itemize}




% This results in what appealing properties and our experiments show this and
% that. High accuracy with less training examples.  

