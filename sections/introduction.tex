




















\section{Introduction}

% Describe the challenge to tackle in a specific context.
% Challenge: hit classification
% Context  : SPI experiments

Single-particle imaging (SPI) by X-ray free-electron lasers (XFELs) is a
promising method of determining the three-dimensional structure of
noncrystalline biomolecules at room temperature.  In SPI experiments,
femtosecond coherent X-ray beams strike biomolecules sprayed into the beam path,
causing scattering of radiation before destruction of the samples inflicted by
the intense X-rays.  This way of collecting scattering datasets is known as the
\textit{diffraction before destruction} approach
\cite{neutzePotentialBiomolecularImaging2000,
chapmanFemtosecondDiffractiveImaging2006,seibertSingleMimivirusParticles2011,
aquilaLinacCoherentLight2015,reddyCoherentSoftXray2017a}.  Scattering patterns
in SPI experiments largely fall into four categories, depending on how X-ray
pulses hit sample particles delivered through jet streams.  Astonishingly, 98\%
X-ray pulses miss their target particles
\cite{shiEvaluationPerformanceClassification2019}, leading to no scattering
pattern identified as a \textit{no-hit}.  When X-ray photons collides with one
and only one sample particle, the emerging scattering pattern is defined as a
\textit{single-hit}. Expectedly, a \textit{multi-hit} happens when X-ray pulses
and a cluster of sample particles intersect. In some cases, X-ray pulses might
hit non-biological objects in delivery medium and those resulting scattering
patterns are labeled as \textit{non-sample-hit}.  Among the four categories,
only \textit{single-hit} images will contribute to reconstruction of electron
density maps through downstream data processing, e.g. orientation recovery and
phase retrieval.  Therefore, the main goal of a SPI hit classifier is to
identify \textit{single-hit}.  


% Past works

Some pioneering works that address the challenge of classifying SPI hits have
been developed.  Unsupervised methods
\cite{yoonUnsupervisedClassificationSingleparticle2011,
giannakisSymmetriesImageFormation2012,schwanderSymmetriesImageFormation2012,
yoonNovelAlgorithmsCoherent2012,
andreassonAutomatedIdentificationClassification2014,
bobkovSortingAlgorithmsSingleparticle2015} focus on extracting features from
image samples and finding clusters of \textit{single-hit} images.  However,
unsupervised solutions are highly problem-specific, that is to say, it's not
realistic to expect \textit{single-hit} images from different biological samples
would cluster in the feature space.  Moreover, despite being unsupervised, those
solutions also require prior knowledge of data distribution in feature space for
classification, rendering it incapable of being automated. On the other hand,
supervised solutions based on artificial neural network models
\cite{shiEvaluationPerformanceClassification2019,
ignatenkoClassificationDiffractionPatterns2021} have been employed to learn from
labelled images and provide predicted classification.  In these neural network
models, convolutional neural networks (CNN) were applied for feature extraction
and a fully connected layer is directly attached to the convolutional layers for
classification.  Training both the CNN feature extraction module and the
classifier simultaneously can be very inefficient and become a bottleneck in
real-time classification.  Furthermore, all previous solutions, unsupervised or
supervised, require human in the loop for labeling before they can be applied to
classification.  It would be great if we can train a model that can be
generalized to recognize hits obtained from various biological samples.  To our
knowledge, there's currently no such hit classifier whose generalizability has
even been demonstrated on simulated hits.  


% (3) What are the core challenges?  In this work, we do this.  Concretely, our
% contributions are ... 

In this work, we strive to tackle the problem of real-time SPI hit
classification by training an artificial neural network model that directly
learns the embedding of hit images.  Depending on whether the solution requires
human in the loop, there are two possible directions to approach this goal: (1)
Use an experiment-specific model but trained efficiently in real time; (2) Train
a generalized model that identifies hits from entirely different biological
samples.  In the first scenario, the primary challenge is the design of a nerual
network model that allows efficient training.  In the second scenario, we face a
more practical challenge: there is no existing experimental or simulated dataset
that contains hits from a large (hundreds or thousands) variety of biological
samples.

Embedding method has been actively studied in the domain of machine learning,
especially for computer vision or natural language processing applications.  One
successful example is FaceNet \cite{schroffFaceNetUnifiedEmbedding2015} that is
a generalized face embedding model with a deep convolutional neural network (CNN)
trained on 100-200 million images.  FaceNet uses a contrastive learning approach
to distinguish one class from another.  Fundementally, the model doesn't try to
learn human-engineered notions of a \textit{single-hit}, but it learns to
understand that \textit{single-hit} is different from \textit{multi-hit} or
\textit{non-sample-hit}.  Our insight is that it is much easier to draw contrast
between two things than to learn the notion of an individual class.  Similar to
the FaceNet model, our neural network model learns hit embeddings directly by
drawing constrasts from many triplets of hits.  A triplet consists of an anchor
image, a positive image that shares the same label as the anchor, and a negative
image that has a different label.  We train our model using the triplet loss
function proposed in the FaceNet.  

Concretely, our contributions are summarized below.  

\begin{itemize}

    \item We introduce a neural network model that classifies SPI
    \textit{single-hit} with a 99\% accuracy but only takes under 15 minutes to
    train on a NVIDIA 1080Ti.  The model learns from 80 uniquely labeled hits
    per class and there are three classes in consideration: \textit{single-hit},
    \textit{multi-hit} and \textit{non-sample-hit}.  Thanks to the low demand on
    the number of training examples and short training time, our model is
    sufficient to take on real-time hit classification tasks.  

    \item We highlight a promising result that our neural network model can be
    generalize to recognize \textit{simulated} SPI hits from different
    biological samples with a 86\% test accuracy.  The practical implication of
    this result is currently limited due to the shortage of experimental
    datasets, but our model is a step closer to realizing a generalized hit
    classification model.

\end{itemize}
