\section{Introduction}

% (1) What is SPI and what problem in SPI have I been tackling?  
% - SPI is an important technique and where it stands in the structural biology toolkits.
%   - Room temperature.
%   - No need for crystalline.
%   - One or two sentences to show why it's important.  
% - SPI scattering patterns are like what, and classification is the problem.

Single-particle imaging (SPI) with X-ray free-electron lasers (XFELs) is a
promising method of determining the three-dimensional structure of
noncrystalline biomolecules at room temperature.   In SPI experiments,
femtosecond coherent X-ray beams strike biomolecules sprayed into the beam path,
causing scattering of radiation before destruction of the samples inflicted by
the intense X-rays.  This way of collecting scattering datasets is known as the
'diffraction before destruction' approach
\cite{neutzePotentialBiomolecularImaging2000,
chapmanFemtosecondDiffractiveImaging2006,seibertSingleMimivirusParticles2011,
aquilaLinacCoherentLight2015,reddyCoherentSoftXray2017}.  Scattering patterns in
SPI experiments largely fall into four categories, depending on how X-ray pulses
hit sample particles delivered through jet streams.  Astonishingly, 98\% X-ray
pulses miss their target particles
\cite{shiEvaluationPerformanceClassification2019}, leading to no scattering
pattern identified as a 'no hit'.  When X-ray photons collides with one and only
one sample particle, the emerging scattering pattern is defined as a 'single
hit'. Expectedly, a 'multi hit' happens when X-ray pulses and a cluster of
sample particles intersect. In some cases, X-ray pulses might hit non-biological
objects in delivery medium and those resulting scattering patterns are labeled
as 'unintended hit'.  Among the four categories, only 'single hit' images will
contribute to reconstruction of electron density maps through downstream data
processing, e.g. orientation recover and phase retrieval.  Therefore, an
efficient SPI scattering pattern classifier is long desired, but building one
has remained to be an elusive task.


% (2) The core challenges in the classification problems are this and that.
% ...Previous work on X has addressed with Y
% 
% - unsupervised method
%   - typically problem-specific
%   - prior knowledge about the location of deisred cluster needs to be known.  
% - supervised method
%   - some requires complete retraining.  
%   - requires fine tune, like the layers of the network.  
%   - still relies on hit finding for binary classifier, whereas classifier doesn't care.  

Some pioneering works that address the challenge of classifying SPI scattering
pattern have been developed.  Unsupervised methods
\cite{yoonUnsupervisedClassificationSingleparticle2011,
giannakisSymmetriesImageFormation2012,schwanderSymmetriesImageFormation2012,
yoonNovelAlgorithmsCoherent2012,
andreassonAutomatedIdentificationClassification2014,
bobkovSortingAlgorithmsSingleparticle2015a} focus on extracting features from
image samples and finding clusters of 'single hit' images.  However,
unsupervised solutions are highly problem-specific, that is to say, it's not
realistic to expect 'single hit' images from different biological samples would
form clusters that overlap in the feature space.  Moreover, despite being
unsupervised, those solutions also require prior knowledge of data distribution
in feature space for classification, rendering it incapable of being automated.
Supervised solutions based on neural network models learn from labelled images
and provide predicted classification
\cite{shiEvaluationPerformanceClassification2019,
ignatenkoClassificationDiffractionPatterns2021}.  Convolution neural networks
(CNN) were employed for feature extraction and a fully connected layer is
directly attached to the convolutional layers for classification.  Training such
networks demands a large sample size for each class, which might not be
available during the course of data collection.  Specifically, the population of
images vary substantially across different classes, resulting in imbalanced
training data detrimental to model training.  


% (3) In this work, we do this.  Concretely, our contributions are ... Siamese
% neural network.  
% Challenges:
% - 

In this work, we strive to approach the goal of classifying scattering patterns
in nearly real-time SPI data collection {\color{gray} with a figure to summarize
what it is}.  

%% We propose a Siamese network for diffraction pattern classification.  

%% {
%% \setlength{\parindent}{4em}
%% \color{gray} 
%% 
%% \indent What is it concretely?  Embedding network, triplet loss function,
%% training strategies, validation strategies, 
%% 
%% }



% (4) This results in what appealing properties 
%     and our experiments show this and that.
% High accuracy with less training examples.  

\

Describe the unqiue characteristics of Siamese network and triplet loss
function.  Our experiments show a high accuracy for diffraction pattern
classification using this model.
