




















\section{Introduction}

% Describe the challenge to tackle in a specific context.
% Challenge: hit classification
% Context  : SPI experiments

Single-particle imaging (SPI) by X-ray free-electron lasers (XFELs) is a
promising method of determining the three-dimensional structure of
noncrystalline biomolecules at room temperature.  In SPI experiments,
femtosecond coherent X-ray beams strike biomolecules sprayed into the beam path,
causing scattering of radiation before destruction of the samples inflicted by
the intense X-rays.  This way of collecting scattering datasets is known as the
\textit{diffraction before destruction} approach
\cite{neutzePotentialBiomolecularImaging2000,
chapmanFemtosecondDiffractiveImaging2006,seibertSingleMimivirusParticles2011,
aquilaLinacCoherentLight2015,reddyCoherentSoftXray2017a}.  Scattering patterns
in SPI experiments largely fall into four categories, depending on how X-ray
pulses hit sample particles delivered through jet streams.  Astonishingly, 98\%
X-ray pulses miss their target particles
\cite{shiEvaluationPerformanceClassification2019}, leading to no scattering
pattern identified as a \textit{no-hit}.  When X-ray photons collides with one
and only one sample particle, the emerging scattering pattern is defined as a
\textit{single-hit}. Expectedly, a \textit{multi-hit} happens when X-ray pulses
and a cluster of sample particles intersect. In some cases, X-ray pulses might
hit non-biological objects in delivery medium and those resulting scattering
patterns are labeled as \textit{non-sample-hit}.  Among the four categories,
only \textit{single-hit} images will contribute to reconstruction of electron
density maps through downstream data processing, e.g. orientation recovery and
phase retrieval.  Therefore, an efficient SPI hit classifier has been long
desired.  


% Past works

Some pioneering works that address the challenge of classifying SPI hits have
been developed.  Unsupervised methods
\cite{yoonUnsupervisedClassificationSingleparticle2011,
giannakisSymmetriesImageFormation2012,schwanderSymmetriesImageFormation2012,
yoonNovelAlgorithmsCoherent2012,
andreassonAutomatedIdentificationClassification2014,
bobkovSortingAlgorithmsSingleparticle2015} focus on extracting features from
image samples and finding clusters of \textit{single-hit} images.  However,
unsupervised solutions are highly problem-specific, that is to say, it's not
realistic to expect \textit{single-hit} images from different biological samples
would cluster in the feature space.  Moreover, despite being unsupervised, those
solutions also require prior knowledge of data distribution in feature space for
classification, rendering it incapable of being automated. Supervised solutions
based on artificial neural network models
\cite{shiEvaluationPerformanceClassification2019,
ignatenkoClassificationDiffractionPatterns2021} have been employed to learn from
labelled images and provide predicted classification.  In these neural network
models, convolution neural networks (CNN) were applied for feature extraction
and a fully connected layer is directly attached to the convolutional layers for
classification.  Training both the CNN feature extraction module and the
classifier simultaneously demands a resonably large sample size for each class.
For example, \cite{shiEvaluationPerformanceClassification2019} labeled 79
\textit{single-hit} images and 121 non-\textit{single-hit} images, whereas
\cite{ignatenkoClassificationDiffractionPatterns2021} labeled 165
\textit{single-hit} images and 390 non-\textit{single-hit} images.  As a result,
hit-labeling will cause a considerable delay before the nerual network models
are trained and deployed, which still remains to be a challenge in real-time
hit classification.  


% (3) What are the core challenges?  In this work, we do this.  Concretely, our
% contributions are ... 

In this work, we strive to tackle the problem of real-time SPI hit
classification by training an artificial neural network model that learns the
embedding of any SPI hit image directly.  There are two possible directions to
reach this goal: (1) Train an experiment-specific model in real time; (2) Train
a generalized model that might work in multiple experiments with entirely
different biological samples.  In the first scenario, the primary challenge is
to demonstrate the applicability of embedding method when there are only small
number of training examples available per class.  In the second scenario, we
face a more practical challenge: there is no existing experimental dataset that
contains a comprehensive set of biological samples.  We think the first scenario
is more significant for it can potentially bring immediate translation into
practices. Meanwhile, we also think there is important academic value in
evaluating the feasibility of training a generalized embedding model for SPI hit
classification.  

Embedding method has been actively studied in the domain of machine learning,
especially for computer vision or natural language processing applications.  One
successful example is FaceNet \cite{schroffFaceNetUnifiedEmbedding2015} that is
a generalized face embedding model with a deep convolutional neural network (CNN)
trained on 100-200 million images.  FaceNet uses a contrastive learning approach
to distinguish one class from another.  Fundementally, the model doesn't try to
learn the notion of a \textit{single-hit}, but it learns to understand
\textit{single-hit} is different from \textit{multi-hit} or
\textit{non-sample-hit}.  Our insight is that it is much easier to draw contrast
between two things than to learn the notion of them.  Therefore, we apply a
FaceNet-like model to tackle the two challenges discussed above.  From a
practical perspective, we want to answer: can a FaceNet-like model learn SPI hit
embedding with a small number of training examples?  Likewise, from a
theoretical persepective, we want to answer: can we achieve a generalized hit
embedding model by training it on thousands of simulated SPI hit images?  


% (4) What will we present to tackle the challenge? Concrete contributions.

Concretely, our contributions are summarized below.  

\begin{itemize}

    \item We introduce a neural network model that learns from a small sized
    dataset, but still achieve a 95\% test accuracy at a precision of 92\% in
    predicting \textit{single hit}. Specifically, the training set has 22
    \textit{single-hit} images, 32 \textit{multi-hit} images and 26
    \textit{non-sample-hit} images.  All examples belong to one single
    experiment with PR772 virus as the biological specimens.  The low demand on
    the size of training examples opens the door to real-time classification
    tasks.  Moreover, the CNN backbone of our model, consisting of only two
    convolutional layers, is computationally lightweight, making it more capable
    for real-time applications.  

    \item We highlight a promising result that our neural network model is
    generalized to recognize simulated SPI hits from different biological
    samples with a 90\% test accuracy.  We choose those biological samples that
    have a molecular weight larger than $380 kDa$.  The reported test accuracy
    is achieved by training our model on $50\,000$ hits sampled from 10\% of the
    unique biological molecules in our curated simulated dataset.  The practical
    implication of this result is currently limited due to the shortage of
    experimental datasets.  Nonetheless, our result shows that a generalized hit
    classification model is possible.

    %% \item We introduce a new large-scale synthetic SPI image database.  The
    %% datasets have accurate labels and diverse scattering patterns as they are
    %% provided by simulating SPI images from {\color{red} 7000+} biomolecules with
    %% molecular weights larger than 380 $KDa$.  

\end{itemize}
