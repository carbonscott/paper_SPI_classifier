

















\newcommand{\RN}[1]{%
  \textup{\uppercase\expandafter{\romannumeral#1}}%
}

\section{Conclusions}

We studied the performance of triplet neural networks on identifying \textit{hit}
s in SPI experiments.  Firstly, we presented strong feasibility of applying our
model to real-time SPI \textit{hit} classification, which can be trained on a
small size dataset under 15 minutes on a moderate GPU like NVIDIA 1080 Ti.
Secondly, we also demonstrated decent generalizability of our model in
predicting \textit{single-hit}s simulated from various PDBs.  This finding
offers opportunities for building a universal \textit{hit} classifier that can
identify \textit{hit}s from distinctive biological samples.  

One major challenge of using our classifier in a real-time application is to
keep up with a potentially much higher detector readout rate.  Our model can
perform classification at a rate of nearly $450 \text{Hz}$ on a NVIDIA 1080 Ti,
which still falls far short to the $1 \text{MHz}$ detector readout rate in
facilities like LCLS-\RN{2} by three orders of magnitude.  Therefore, we propose
to incorporate \textit{hit}-filtering by intensity thresholding as the first
step, which can quickly eliminate the vast majority of ``missed-target'' shots.
Then, we follow up the \textit{hit} classification with our neural network based
classifier.  
